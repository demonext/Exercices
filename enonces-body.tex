\begin{document}

\title{Exercices de morphologie dérivationnelle}
\author{}
\date{Étudiant : \student}

\maketitle

\section{Verbes en \emph{-iser}}

Les exemples suivants illustrent quatre types de verbes dont l'infinitif  se termine par la séquence \emph{-iser}.

\begin{exe}
  \ex\begin{xlist}
    \ex \adjexun
    \ex \nounexun
    \ex \label{ex:exemple1} \prefixexun
    \ex \simplexexun
  \end{xlist}
\end{exe}

\begin{enumerate}
\item Pour chacun des exemples suivants, relève-t-il d'un des types ci-dessus, et si oui, lequel ?
  \begin{exe}
    \ex \aclasserexun
  \end{exe}
\item Expliquez ce qu'il y a de commun à tous les exemples d'une même liste.
  
\item Décrivez les éventuelles différences entre exemples d'une même liste.
  
\item Écrivez une description formelle de la règle ou des règles nécessaires pour dériver les exemples en~(\ref{ex:exemple1}), soit sous la forme d'une entrée lexicale de morphèmes, soit sous la forme d'un appariement de schémas lexicaux. Si certains aspects vous semblent impossibles à formaliser, expliquez pourquoi.
  
\end{enumerate}

\section{Noms en \emph{-ité}}

Les exemples suivants illustrent cinq types de noms se terminant par la séquence \emph{-té}.

\begin{exe}
  \ex
  \begin{xlist}
    \ex \absurditeexdeux 
    \ex \label{ex:exemple2} \adaptabiliteexdeux
    \ex \biodiversiteexdeux
    \ex \bonteexdeux
    \ex \pariteexdeux
  \end{xlist}
\end{exe}

\begin{enumerate}
\item Pour chacun des exemples suivants, relève-t-il d'un des types ci-dessus, et si oui, lequel?
  \begin{exe}
    \ex \aclasserexdeux
  \end{exe}
\item Expliquez quelles sont les différences morphologiques entre ces listes.
  
\item Écrivez une description formelle de la règle morphologique qui est à l'œuvre en~(\ref{ex:exemple2}), soit sous la forme d'une entrée lexicale de morphèmes, soit sous la forme d'un appariement de schémas lexicaux.
\end{enumerate}

\section{Verbes dénominaux affixés}

Les exemples suivants illustrent trois types de relations de dérivation formant des verbes sur base nominale. 

\begin{exe}
  \ex \label{ex:exemple31}
  \begin{xlist}
    \ex \panifierextrois  
    \ex  \embrasserextrois
    \ex \vampiriserextrois
  \end{xlist}
\end{exe}

Chaque nom des exemples ci-dessus est caractérisé par un type sémantique, cf. ci-dessous.


\begin{exe}
\ex \label{ex:exemple32} \begin{xlist}
\ex \painextroisbis
\ex \label{ex:exemple3}\brasextroisbis
\ex \vampireextroisbis
\end{xlist}
\end{exe}

\begin{enumerate}
\item Pour chaque verbe dérivé de (\ref{ex:exemple31}), 
  \begin{enumerate}
  \item donnez sa  définition en vous servant du nom de base.
  \item Décrivez la construction morphologique complète du verbe en fonction de l'exposant de la règle et du type sémantique du nom de base (\ref{ex:exemple32}).
  \end{enumerate}   
\item Généralisez vos observations au moyen d'une description formelle de la règle morphologique qui est à l'œuvre en~(\ref{ex:exemple3}), soit sous la forme d'une entrée lexicale de morphèmes, soit sous la forme d'un appariement de schémas lexicaux.
\item D'après les exemples de (\ref{ex:exemple31}), est-ce qu'à votre avis le type ontologique du nom de base détermine toujours le contenu sémantique du verbe? Si votre réponse est non, justifiez votre réponse par un exemple. 
  
\end{enumerate}

\section{Relation de conversion verbe/nom}

Les exemples suivants illustrent trois types de relations de dérivation entre un verbe et un nom. 

\begin{exe}
  \ex \label{ex:exemple41}
  \begin{xlist}
    \ex \zigzaguerexquatre  
    \ex  \vitriolerexquatre
    \ex \ventriloquerexquatre
  \end{xlist}
\end{exe}

Chaque nom des exemples ci-dessus est caractérisé par un type sémantique, cf. ci-dessous.

\begin{exe}
  \ex \label{ex:exemple42}
  \begin{xlist}
    \ex \zigzagexquatrebis
    \ex \vitriolexquatrebis
    \ex \label{ex:exemple4} \ventriloqueexquatrebis
  \end{xlist}
\end{exe}


\begin{enumerate}
\item Pour chaque verbe de (\ref{ex:exemple41}), 
  \begin{enumerate}
  \item Donnez sa  définition en vous servant du nom de base.
  \item Décrivez l'ensemble des propriétés de la relation de conversion  verbe / nom, en particulier en fonctio du type sémantique du nom de base (\ref{ex:exemple42}.
  \end{enumerate}   
\item Généralisez vos observations au moyen d'une description formelle de la règle morphologique qui est à l'œuvre en~(\ref{ex:exemple4}), soit sous la forme d'une entrée lexicale de morphèmes, soit sous la forme d'un appariement de schémas lexicaux.
\item D'après les exemples de (\ref{ex:exemple41}), est-ce qu'à votre avis le type ontologique du nom de base détermine toujours le contenu sémantique du verbe? Justifier votre réponse.
\end{enumerate}

\section{Histoires dérivationnelles}

Les paires de lexèmes suivants illustrent 3 types d'histoires dérivationnelles qui connectent un lexème complexe  (premier élément de la paire) à son ascendant (deuxième élément de la paire). La relation entre les deux comporte plus d'une étape dérivationnelle. 

\begin{exe}
  \ex \label{ex:exemple51}\begin{xlist}
    \ex \cogestionnaireexcinq
    \ex  \poulaillerexcinq
    \ex \langagierexcinq
  \end{xlist}
\end{exe}

\begin{enumerate}
\item \label{it:question-ex5} Décrivez l'histoire dérivationnelle de chaque lexème complexe de~(\ref{ex:exemple51}), de manière à le relier au lexème plus simple correspondant.  
  \begin{enumerate}
  \item Pour chaque étape d'analyse que vous identifiez, décrivez les propriétés formelles, catégorislles et sémantiques de la relation.
  \item Montrez que la relation est reproductible en donnant l'exemple d'un autre couple de lexèmes dont la relation dérivationnelle vérifie les mêmes propriétés.
  \end{enumerate}
\end{enumerate}

Examinez maintenant la paire de lexèmes suivante, qui illustre un 4\ieme\ cas d'histoire dérivationnelle :

\begin{exe} 
  \ex \label{ex:exemple52} \bitumisationexcinq
\end{exe}


\begin{enumerate}
\item Même question qu'en (\ref{it:question-ex5}) avec le premier lexème de (\ref{ex:exemple52}).  Montrez qu'ici, entre le lexème plus complexe et le lexème plus simple, la trajectoire n'est pas directe, comme elle l'est avec les cas précédents : le lexème plus simple n'est pas un ancêtre direct de l'autre lexème.
\end{enumerate}

\section{Nouvelles histoires dérivationnelles}

Les deux paires de lexèmes de l'exemple (\ref{ex:exemple6})  illustrent chacune un type d'histoire dérivationnelle qui connecte un lexème à la fois préfixé et suffixé  (premier élément de la paire) à un lexème plus simple de sa famille dérivationnelle  (deuxième élément de la paire). La relation entre les deux comporte plus d'une étape dérivationnelle. 

\begin{exe}
  \ex \label{ex:exemple6}
  \begin{xlist}
    \ex \label{ex:exemple61} \antigrippalexsix
    \ex \label{ex:exemple62} \postglaciaireexsix
  \end{xlist}
\end{exe}

\begin{enumerate}
\item Décrivez l'histoire dérivationnelle de chaque lexème complexe de~(\ref{ex:exemple6}), de manière à le relier au lexème plus simple correspondant.  
  \begin{enumerate}
  \item Pour chaque étape d'analyse que vous identifiez, décrivez les propriétés formelles, catégorislles et sémantiques de la relation.
  \item Montrez que la relation est reproductible en donnant l'exemple d'un autre couple de lexèmes dont la relation dérivationnelle vérifie les mêmes propriétés.
  \end{enumerate}
\item Quelles différences d'analyse remarquez-vous entre (\ref{ex:exemple61}) et (\ref{ex:exemple62}) ?
\end{enumerate}

\section{Compléter la famille dérivationnelle}

Les deux paires de lexèmes de l'exemple (\ref{ex:exemple7})  constituent chacune une famille dérivationnelle minimale. 

\begin{exe}
  \ex \label{ex:exemple7}
  \begin{xlist}
    \ex \label{ex:exemple71} \applicateurexsept
    \ex \label{ex:exemple72} \contestataireexsept
  \end{xlist}
\end{exe}

\begin{enumerate}
\item Décrivez les propriétés formelles, catégorielles et sémantiques de la relation morphologique entre les deux lexèmes de (\ref{ex:exemple71}) et (\ref{ex:exemple72}) 
\item Complétez chacune des familles (\ref{ex:exemple71}) et (\ref{ex:exemple72})  au moyen de deux membres supplémentaires.
  \begin{enumerate}
  \item Décrivez les propriétés formelles, catégorielles et sémantiques des nouvelles relations.
  \end{enumerate}
\item Trouvez une famille de 4 lexèmes dont les membres entretiennent entre eux les mêmes relations morphologiques que les membres de la famille  (\ref{ex:exemple71}), de sorte que les deux familles se superposent et forme un paradigme.
\item Même chose avec la famille  (\ref{ex:exemple72}) : trouvez une famille de 4 lexèmes dont les membres entretiennent entre eux les mêmes relations morphologiques de sorte que les deux familles se superposent et forme un paradigme.
\end{enumerate}

\section{Variations formelles en dérivation}

L'exemple (\ref{ex:exemple8})  comporte un ensemble de noms qui se terminent par \emph{-at} et qui sont tous des variantes du même dérivé du  nom qui occupe la dernière position dans la liste. 

\begin{exe}
  \ex \label{ex:exemple8}\exhuit
\end{exe}

\begin{enumerate}
\item Décrivez en quoi les différentes variantes sont différentes.
\item Proposez une analyse (ou plusieurs) pour expliquer chaque variante : est-elle motivée phonologiquement ? L'allomorphie concerne-t-elle le radical ou l'exposant ?
\end{enumerate}

% =================================================================

\section{Décalages sens-forme}

Examinez les couples de lexèmes de l'exemple (\ref{ex:exemple9}).

\begin{exe}
  \ex \label{ex:exemple9}
  \begin{xlist}
    \ex \cartesianismeexneuf
    \ex \urbanismeexneuf
    \ex \historicismeexneuf
    \ex \cellularismeexneuf
  \end{xlist}
\end{exe}

\begin{enumerate}
\item Pour la relation entre le complexe (à gauche) et le simple (à droite) dans chaque paire, a-t-on affaire :
  \begin{enumerate}
  \item à une ou plusieurs étapes dérivationnelles ?
  \item à une variation radicale ?
  \item à une relation motivée sémantiquement (et corrélée à une relation intermédiaire motivée formellement) ?
  \end{enumerate}
\end{enumerate}

\end{document}
